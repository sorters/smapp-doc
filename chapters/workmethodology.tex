In order to follow and measure the progress of the project, a series of goals will be set along a timeline. These goals will derive on a first basis from the basic features described on the scope of the project. These goals will be iterative and work in a similar way as the sprints in the Scrum methodology:

\begin{itemize}
\item Each goal will be broken down and translated to smaller tasks, until these tasks become atomic enough as to be easily and directly measurable.

\item Each of these tasks will include an estimation of their cost regarding time in order to get an idea of the whole timeline. This should not be used as something to enforce in a rigid way, but to provide a way to monitor the progress. 

\item Progress will be measured by the number of the tasks that are considered to be completed. 
\end{itemize}

The project will be successful if all the tasks are completed and therefore all the goals are covered.

As this project will be developed by only one person, at least on the first phase (as described on the scope of the project) there is no need for collaborative work tools.

Regarding planning and success tracking, Trello will be used, although the possibility of using JIRA could be explored.

\section{Validation}
Since the working methodology for the development of this project will be TDD based, the validation of the tasks can be closely related to the software passing correctly the written tests for every use case. 

Therefore there will be an objective way of measuring the success of each development phase: from each requirement a set of use cases will be derived, and for each of these use cases, several features can be specified. In order to be able to implement these features, a test suite for each one has to be written. The task is successful if all tests pass correctly.

As soon as a feature is completed, and therefore a use case is covered, and finally a goal is reached, these phases can be transferred from a ‘Doing’ section to the ‘Done’ section in Trello or in JIRA, depending on the tool that ends up being used.

The project can be considered finished once all the tasks (and by extension all the goals) are ‘Done’.