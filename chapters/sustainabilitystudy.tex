The sustainability aspect of this project is presented from the following perspectives:

\section{Economic}
\label{sec:131}
The economic study of this project is covered by the budget and cost analysis in the previous sections of this document, which include the total cost and its breakdowns during the whole analysis and development phase. After this phase, the costs associated to the evolution, if any, of the project, can be considered negligible, as all the modifications and support on the product will be in the form of community contributions, which are volunteer. This project will not generate profit or revenue in a direct manner, so it’s not meant to be a competitive product. However, with the advantages of being free and expected to be really easy to use and adapt, it is expected that the product will gather community and a user base.

Regarding the cost levels, it would probably be quite difficult to develop a similar platform with a lower economic cost, as only one person will be in charge of the task. However, the downside of this is that the project will take much longer than if it had a proper development team like a company or a startup could provide. That said, a lot of previous knowledge and experience will go to making the development phase more efficient, as the core developer has gained plenty proficiency in the technologies and tools used throughout the project. After that, further improvements are expected to be contributed by proficient and experienced people from the open-source community.

Given the very low cost and the potential positive impact, the Economic perspective can be valuated with an 8. 

\section{Social}
\label{sec:132}
The goal of this project is to provide an improvement in the tools and platforms used by small companies and stores that cannot afford, or don’t need, or lack the expertise to use more complex and expensive available platforms.

Therefore, if the product is successful, it will be a significant benefit for users that until now have not been able to correctly and precisely track and manage their stocks and products. This will mean, for them, a potential improvement on efficiency and cost management that can result in an increase on revenue. The only disadvantage this product can represent is for other similar platforms (paid, or free but limited) that could see a decrease on their users, if these decide to migrate and start using this new platform.

The small negative consequences for big companies compared to the potentially big impact for smaller users justifies a high score in this perspective, and the valuation for this perspective is an 8.

\section{Environmental}
\label{sec:133}
This project doesn’t have any significant environmental effects, either positive or negative. During the development phase, all of its material resources will be shared with ongoing projects or tasks and so it will not cause an increase in any kind of spending. After that, being the product a very lightweight solution, it is not expected that it will generate any significant consumption of additional hardware, power or any other resource that could affect the environment in a negative way. The solution is expected to work properly on a simple desktop PC based server, and most, if not all, small companies and other users will already be using one. The low impact the project has in this aspect could mean a high valuation for this section. However, even though the project’s influence will not worsen the imprint on the environment, it is not likely that it will improve it significantly. The valuation will therefore be slightly positive, with a 6.

\section{Conclusion}
The conclusion of this study is, that given the particular features of this project, no negative effects or outcomes can be foreseen for the environment, while it could mean a big social and economic improvement for its users, small companies and business owners, with small negative effects on other big and powerful platforms that provide similar services.
\hfill\break
\begin{center}
    \begin{tabular}{ | c | c | c | c | c |}
    	\hline
 		\textbf{Sustainability} & Economic & Social & Environmental & \textbf{Total} \\ \hline
 		Justification & \hyperref[sec:131]{13.1} & \hyperref[sec:132]{13.2} & \hyperref[sec:133]{13.3} & - \\ \hline
 		\textbf{Valuation} & 8 & 8 & 6 & \textbf{22} \\ \hline
    \end{tabular}
\end{center}