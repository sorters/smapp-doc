Unfortunately there is not much time or room for variations in this planning, because of time constraints. However, at the end of the plan there is a two-week buffer that allows room for corrections in case some of the phases took longer than expected.

It has also to be noted that, although the time estimates for each phase and their parts have been estimated as accurately as possible, some may take more or less. Herein lies the advantage of using an agile methodology, which allows adjusting the schedule and planning more or less at real time, so to say. The initial intention of the plan would be to compensate one time deviation with another one. If one task takes less time than expected, this will allow more margin for other tasks that may take longer, and in reverse.

An additional advantage of using an agile methodology throughout the project is that having more milestones than, for instance, the waterfall paradigm, it is more easy to quickly detect and correct a deviation on the time plan. If deviations like these occur in the middle of the project, measures can be taken on a short notice, and if the deviation is corrected, there is no need to modify the rest of the time plan.

Therefore the idea is to always respect the deadline and compensate during time project, and use the final two-week buffer only as a last resort.