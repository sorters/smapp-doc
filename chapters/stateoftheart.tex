Particularly in this area, there is a really vast amount of software solutions already available, both proprietary and open-source, with paid and free versions, that allow more or less features accordingly. Some of these solutions belong to industry giants long consolidated in their sector, such as SAP and ERP, while others are minor or specialized solutions based on a small scale local market.
While it is good to know about real world use cases and successful solutions, this analysis pretends to focus on the key points these solutions address in order to fulfill each need. In some cases, an actual solution may be mentioned, but due to the massive number of software packages that would be available as an example for each point, it is preferred to overlook them in order to avoid creating noise.

Following are the key points that define a modern stock management solution \cite{3}. These points are now described in a basic and essential manner, but will be explored more in-depth in a technical manner later in this document.

\section{Counting and quantifying}
The main function of a stock management solution is to have accurate and comprehensive accounts of the stock, and allow interaction with this data. In short, therefore, a solution has to be able to keep a record of a series of items, which comprise the stock itself, and their current quantities.
Upon that, the solution should be able to modify these items, in order to add or remove them, or to change their behaviour and identity if needed.
Furthermore, it should allow to modify the quantities. In general terms a decrease in a quantity would mean a sale, while, in order to increase a quantity, an order should be placed. Usually a stock management software solution will include support for both features.
\section{Ordering}
This can also be known as “warehouse flow” in the literature. It is important to have a complete understanding of the stock, not only regarding quantities, but also regarding categories, types of product, etc, at an abstract or conceptual level. The more comprehensive the management of a stock is, the bigger the impact it has on efficiency and reduction of costs. Therefore features that allow the agile sorting and classification of stock become important.
\section{Reporting}
It is not enough that the software solution is aware of the current state of the stock. The solution must provide an easy and user-friendly way for the people who use it to get a comprehensive and thorough understanding of the state of the stock, in order to control it and react to possible situations and opportunities. Therefore the solution must be able to set and provide significant KPIs.
Regarding this area, alongside the native features that many stock management solutions provide, there are plenty of reporting, analytics and dashboard solutions that can integrate easily with this kind of software. Some of these solutions will be later explored in the project design.
\section{Demand forecasting}
One of the main areas where the efficiency of a stock management system is decided is the one regarding demand forecasting. This concern \cite{4}, addressed perhaps most notably by Toyota in their production chains in Japan during the 1960s and 1970s bases its logic in the principle of stock availability optimization based on the needs of a particular moment.
The paradigm of a warehouse changes from being a platform of massive long-term storage to being a smaller, optimized and temporary housing for a moving good or product. This leads to new principles along the lines of “make only what you can sell”, and “sell only what you can make” \cite{5}. Therefore, software solutions that can provide accurate estimates of both what you can sell and what you can make (and of course, taking into account what you can store) gain a lot of protagonism, and so, along with the recent rise of big data and particularly of data science paradigms many software solutions apply heuristic and trend-aware algorithms in order to provide accurate estimations of the effects from external factors related to the stock, such as increases or decreases in demand, seasonality peaks, etc.
\section{Stock rotation}
Another aspect of stock management is stock rotation. This is particularly important with any stocks containing perishable goods, most notably in the food industry, and also with pharmaceutical and chemical products. These types of stock, having a limited storage life regardless of being sold or not, have to be recurrently renovated. Again, this is reflected in current software solutions in many ways, with features such as periodic order placements. Some of these solutions, like SAP or ERP, include features specific to this concern.
\section{Accessibility}
Perhaps accessibility is not as as obvious a feature as the previously mentioned. After all, one could think that the idea is to have the management of the stock near to, or at the same place as the stock itself. But with the increase in globalization over the last decades and, more recently, the paradigm shift to mobile-first philosophy, easy access to information and tools has become a priority. For instance, particularly in the case of stock management software, ease of integration with handheld devices is a must, alongside multi-platform support of the solution and ease of access from wherever a user might be.